\section{Task 3}

\begin{question}
   Create and update a materialized view with \textbf{CREATE MATERIALIZED VIEW} and
   \textbf{CREATE MATERIALIZED VIEW LOG} in \textbf{ORACLE}.\\
   Consider the following frequent queries of interest:
   \begin{itemize}
       \item
       Separately for each ticket type and for each month analyze the average daily
       revenue.
       \begin{lstlisting}[language=SQL]
            select s.ticketType, t.month, t.year,
                sum(s.totalAmount) / count(distinct t.validDate) as AVGDailyRevenue
            from TicketSale s, TimeDim t
            where s.timeid = t.timeid
            group by s.ticketType, t.month, t.year
            order by t.month, s.ticketType;
       \end{lstlisting}
       \item
       Separately for each ticket type and for each month analyze the cumulative revenue
       from the beginning of the year.
       \begin{lstlisting}[language=SQL]
            select s.tickettype, t.month, t.year,
                    sum(sum(s.totalAmount))
                        over (partition by t.year order by t.month, t.year) as AccRevenue
            from TicketSale s, TimeDim t
            where s.timeid = t.timeid
            group by s.tickettype, t.month, t.year
            order by s.tickettype, t.year;


       \end{lstlisting}
       \item
       Separately for each ticket type and for each month analyze the total number of
       tickets, the total revenue and the average revenue.
       \begin{lstlisting}[language = SQL]
            select s.tickettype, t.month,
                sum(s.numberOftickets),
                sum(s.totalAmount),
                sum(s.totalAmount) / count(distinct t.validDate)
            FROM TicketSale s, Timedim t
            WHERE s.timeid = t.timeid
            Group by s.tickettype, t.month
       \end{lstlisting}
       \pagebreak
       \item
       Separately for each ticket type and for each month analyze the total number of
       tickets, the total revenue and the average revenue for year 2021.
       \begin{lstlisting}[language = SQL]
            select s.tickettype, t.month,
                sum(s.numberOftickets),
                sum(s.totalAmount),
                sum(s.totalAmount) / count(distinct t.validDate)
            FROM TicketSale s, Timedim t
            WHERE s.timeid = t.timeid and t.year = 2021
            Group by s.tickettype, t.month;
       \end{lstlisting}
       \item
       Analyze the percentage of tickets related to each ticket type and month over the
       total number of tickets of the month for each ticket type.
       \begin{lstlisting}[language = SQL]
            select s.tickettype, t.month,
                sum(s.numberOfTickets)/sum(sum(s.numberOfTickets))
                    over (partition by t.month) monthTicketsSoldPercentage
            FROM TicketSale s, Timedim t
            WHERE s.timeid = t.timeid
            Group by s.tickettype, t.month;
       \end{lstlisting}
   \end{itemize}
\end{question}

\pagebreak

\subsection{3.1: Materialized view}

\begin{question}
    Define a materialized view with CREATE MATERIALIZED VIEW useful to
    reduce the response time of the reported frequent queries.
\end{question}

\begin{answer}
    The most comprehensive query statement for the MV would be:
    \begin{lstlisting}[language = SQL]
        select s.ticketType, t.month, t.year,
            sum(s.totalAmount) / count(distinct t.validDate) as AVGDailyRevenue,
            sum(sum(s.totalAmount))
                    over (partition by t.year order by t.month, t.year) as AccRevenue,
            sum(s.numberOftickets),
            sum(s.totalAmount),
            sum(s.totalAmount) / count(distinct t.validDate),
            sum(s.numberOfTickets)/sum(sum(s.numberOfTickets))
                over (partition by t.month) monthTicketsSoldPercentage
        from TicketSale s, TimeDim t
        where s.timeid = t.timeid
        group by s.ticketType, t.month, t.year
        order by t.month, s.ticketType;
    \end{lstlisting}
    However, in order to have a tradeoff between the MV query and the specific queries. The statement would be:
    \begin{lstlisting}[language = SQL]
        select s.ticketType, t.month, t.year,
            sum(s.totalAmount) as totalAmount,
            count(distinct t.validDate) as daysInTheMonth,
            sum(s.numberOftickets) as totalNumberOfTickets
        from TicketSale s, TimeDim t
        where s.timeid = t.timeid
        group by s.ticketType, t.month, t.year
        order by t.month, s.ticketType;
    \end{lstlisting}
    The complete sql statement is:
    \begin{lstlisting}[language = SQL]
        CREATE MATERIALIZED VIEW ticketSaleMV
        BUILD IMMEDIATE
        REFRESH FAST ON DEMAND
        ENABLE QUERY REWRITE
        AS
            select s.ticketType, t.month, t.year,
                sum(s.totalAmount) as totalAmount,
                count(distinct t.validDate) as daysInTheMonth,
                sum(s.numberOftickets) as totalNumberOfTickets
            from TicketSale s, TimeDim t
            where s.timeid = t.timeid
            group by s.ticketType, t.month, t.year;
    \end{lstlisting}

\end{answer}

\subsection{3.2: Materialized view logs}

\begin{question}
    Define the materialized view logs with CREATE MATERIALIZED VIEW LOG for
each table where you deem it necessary. For which tables is it useful to keep track
of logs? Identify all and only the necessary tables. Furthermore, for each table
identify all and only the attributes for which it is necessary to keep track of the
variations.
\end{question}

\begin{answer}
    For the TicketSale table we need: timeId, totalAmount, and number of tickets:
    \begin{lstlisting}[language = SQL]
        create materialized view log on TicketSale
            with sequence, rowid
            (timeid, ticketType, totalAmount, numberOfTickets)
            including new values;
    \end{lstlisting}
    For the TimeDim table we need timeid, month and year:
    \begin{lstlisting}[language = SQL]
        create materialized view log on TimeDim
            with sequence, rowid
            (timeid, month, year)
            including new values;
    \end{lstlisting}
\end{answer}

\subsection{3.3: Updates on materialized view}
\begin{question}
    Specify which operations (e.g. INSERT on a specific table) cause an update of
the defined MATERIALIZED VIEW
\end{question}

\begin{answer}
    INSERT and UPDATE operations on tables TimeDim and TicketSale would update the materialized view TicketSaleMV.
    Specifically the materialized view is updated by the UPDATE operation on table TimeDim the fields month and year,
    and on table TicketSale the fields totalAmount, and number of tickets.
\end{answer}

\pagebreak