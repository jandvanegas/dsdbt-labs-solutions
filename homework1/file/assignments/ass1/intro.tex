\section{Prompt}
Data analysts of the National Association of Italian Museums are interested in analyzing the
average revenue per ticket. In particular, they would like the analyses to address the following features.
\begin{itemize}
\item 
A museum has a unique name, and it is located in a specific city. The province and region
are also stored. The same city can host different museums. Each museum belongs to a
specific category (e.g., “Art”, “Historic sites”, “Natural history”).
\item
A museum may have some additional services available for its public. The systems record
which services are available for each museum. Examples of additional services are “guided
tours”, “audio guides”, “wardrobe”, “café”, “Wi-Fi”. The number of additional services is 10
and their complete list is known.
\item
The tickets sold by each museum are recorded. There are 3 different types of tickets: “Full
Revenue”, “Reduced-student” (for students from 14 to 24 years old) and “Reduced-junior”
(for young people less than 14 years old).
\item
The systems also store how the ticket is purchased. A ticket can be purchased in three
modalities: online, in authorized ticket offices, or directly at the entrance of the museum.
\item
The analyses must be carried out considering the date, month, bimester, trimester,
semester, year, if the date is a working day or a holiday, and time slot of the ticket validity
date. The time slot is stored in 3 ranges of 4-hour blocks (08:00-12:00, 12:01-16:00, 16:01-
20:00). 
\end{itemize}



