\section{Task 4}

\begin{question}
    Update and management of views via Trigger.\\
    Assuming that the CREATE MATERIALIZED VIEW command is not available, create the
    materialized views defined in the previous exercise and define the update procedure
    starting from changes on the fact table created by means of a trigger.
\end{question}

\subsection{4.1: Manual materialized view creation}

\begin{question}
    Create the structure of the materialized view with CREATE TABLE VM1 (…)
\end{question}
    \begin{lstlisting}[language = SQL]
        create table manual_mv(
            ticketType varchar(16) not null,
            month varchar(16) not null,
            year numeric not null,
            totalAmount Decimal not null,
            daysIntheMonth Integer not null,
            numberOfTickets Integer not null,
            constraint pk_manualmv primary key (ticketType, month)
        )
    \end{lstlisting}
\begin{answer}
\end{answer}

\subsection{4.2: Loading data in the materialized view}

\begin{question}
     Specify an example of statement to populate the VM1 table with the necessary
records using the statement:
    \begin{lstlisting}[language = SQL]
         INSERT INTO VM1 (…)
             ( SELECT …
             … )
    \end{lstlisting}

\end{question}

\begin{answer}
    \begin{lstlisting}[language = SQL]
insert into manual_mv
    (ticketType, month,   year, totalAmount, daysInTheMonth, numberOfTickets)
    (select s.ticketType, t.month, t.year,
        sum(s.totalAmount) as totalAmount,
        count(distinct t.validDate) as daysInTheMonth,
        sum(s.numberOftickets) as totalNumberOfTickets
    from TicketSale s, TimeDim t
    where s.timeid = t.timeid
    group by s.ticketType, t.month, t.year);
    \end{lstlisting}
\end{answer}

\subsection{4.3: Trigger operation for manual materialized view}
\begin{question}
    Write the triggers necessary to propagate the changes (insertion of a new record)
made in the FACTS table to the materialized view VM1.
\end{question}
\begin{answer}
    \begin{lstlisting}[language = SQL]
create trigger insertTicketSale
    after insert on TicketSale
    for each row

declare
    recordMonth varchar(16); recordsCount number; newTotalAmount number; newDaysInTheMonth number; newTotalNumberOfTickets number;

Begin

select month into recordMonth  from TimeDim where timeId = :NEW.timeID;

select count(*) into recordsCount from manual_mv where month = recordMonth and ticketType = :NEW.ticketType;

if recordsCount > 0 then

select sum(s.totalAmount), count(distinct t.validDate), sum(s.numberOftickets)
    into newTotalAmount, newDaysInTheMonth, newTotalNumberOfTickets
from TicketSale s, TimeDim t
where s.timeid = t.timeid and s.ticketType=:NEW.ticketType and t.month=recordMonth;

update manual_mv
set totalAmount = newTotalAmount, daysInTheMonth = newDaysInTheMonth, numberOfTickets = newTotalNumberOfTickets
    where month = recordMonth and ticketType = :NEW.ticketType;

else

insert into manual_mv
    (ticketType, month,   year, totalAmount, daysInTheMonth, numberOfTickets)
    ( select s.ticketType, t.month, t.year, sum(s.totalAmount) as totalAmount, count(distinct t.validDate) as daysInTheMonth, sum(s.numberOftickets) as numberOfTickets
        from TicketSale s, TimeDim t
        where s.timeid = t.timeid and ticketType=:NEW.ticketType and t.month=recordMonth
        group by s.tickettype, t.month, t.year
    );
end if; end;
    \end{lstlisting}
\end{answer}
\subsection{4.4: Details about trigger}
\begin{question}
     Specify which operations (e.g. INSERT) trigger the trigger created in 4.3.
\end{question}

\begin{answer}
    The trigger operation done on the item 4.3 is triggered by any INSERT on table TicketSale.
\end{answer}
