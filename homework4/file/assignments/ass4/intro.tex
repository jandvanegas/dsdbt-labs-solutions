\section{Prompt}

\par
The “bike stations” dataset is available, containing information about 65
stations of a bike sharing service. To carry out the homework, it is
recommended to import the dataset as a collection on a database on a local
MongoDB server, using the mongoimport tool.

\begin{tcolorbox}[width=\textwidth, title={Note},]
  To use locally MongoDB, install the “MongoDB Community Edition” and
  refer to the official
  documentation.
\end{tcolorbox}
\begin{tcolorbox}[width=\textwidth, title={Note},]
  To use the mongoimport tool, install “MongoDB Database Tools”,
  (MongoDB Database Tools).
\end{tcolorbox}

\par
You can refer to the tutorial and material provided in the MongoDB lab for
the commands for importing a
collection, running a MongoDB server locally and accessing the MongoDB
command-line interface (shell).

\par
An example of a station extracted from the collection is reported below.

\begin{lstlisting}[language=js]
{
  "_id": ObjectId( "61b75b13fd4d2d1ea82e75f4" ),
  "empty_slots": 10,
  "extra": {
    "number": 57,
    "reviews": 222,
    "score": 4,
    "status": "online",
    "uid": "307"
  },
  "free_bikes": 4,
  "id": "bfa12cb895ac0d7392dde60b6b433cdf",
  "name": "San Francesco da Paola",
  "timestamp": "2021-12-10T14:54:39.185000Z",
  "location": {
    "type": "Point",
    "coordinates": [ 45.068617, 7.689097 ]
  }
}
\end{lstlisting}

\par

To answer the homework questions, it is necessary to report:
\begin{itemize}
  \item The query used to obtain the answer to the question (the query must
  extract only the fields necessary to
  answer the question)
  \item The result of the question
\end{itemize}