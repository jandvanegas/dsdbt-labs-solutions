\section{Prompt}

The following relations are given (primary keys are underlined):

\begin{lstlisting}[language = SQL, escapechar = \%]
    ACCOMODATION(%\underline{CodA}%, NumberOfGuests, Address, City, Region)
    SERVICE(%\underline{CodS}%, ServiceName, ServiceType)
    ACCOMODATION-HAS-SERVICE(%\underline{CodA, CodS}%)
    USER(%\underline{CodU}%, FirstName, Surname, BusinessAccount, BirthDate, Address, City, Region)
    BOOKING(%\underline{CodA,StartDate}%, CodU, EndDate)
\end{lstlisting}

Assume the following cardinalities:
\begin{ttfamily}
\begin{itemize}
    \item card(ACCOMODATION) =$10^5$ tuples, \\
    distinct values of Region = 20
    \item card(SERVICES)=$10^2$ tuples, \\
    distinct values of ServiceType = 20
    \item card(ACCOMODATION-HAS-SERVICE)=$10^6$ tuples,
    \item card(USER)=$10^4$ tuples,\\
    \lstinline{min(date(BirthDate)) = 1/1/1930},\\
    \lstinline{max(date(BirthDate)) = 31/12/2009},\\
    distinct values of Region = 20, \\
    distinct values of BusinessAccount = 2 (“True”, “False”)
    \item card(BOOKING)=$10^7$ tuples, \\
    \lstinline{min(date) = 1/9/2017},  \\
    \lstinline{max(date)) = 31/08/2020}
\end{itemize}
\end{ttfamily}
Furthermore, assume the following reduction factor for the group by condition:
\begin{itemize}
    \item \lstinline{having count(distinct StartDate) > 1} $≈$ 1/10
\end{itemize}

\pagebreak

Consider the following SQL query
\begin{lstlisting}[language=SQL]
    select A.CodA, count(Distinct StartDate)
    from SERVICE S, ACCOMODATION-HAS-SERVICE AHS,
        ACCOMODATION A, BOOKING B, USER U
    where S.CodS=AHS.CodS and A.CodA=AHS.CodA
        and U.CodU=B.CodU and B.CodA=A.CodA

        and (S.ServiceType=”Parking” or S.ServiceType=”domestic appliances”)
        and A.Region=”Liguria”
        and B.StartDate>=1/5/20
        and B.StartDate<=31/8/20
        and U.Region<>”Piemonte”

    group by CodA

    having count(distinct StartDate)>1
\end{lstlisting}